 \documentclass[ngerman,
 				a4paper,
 				fontsize=12pt]
 				{article}

\usepackage[a4paper, left=22mm, right=22mm, top=22mm, bottom=22mm]{geometry}
\usepackage[hidelinks]{hyperref}
\usepackage[utf8]{inputenc}
\usepackage{graphicx}
\usepackage{wrapfig}
\usepackage{xcolor}

\title{Algorithmen und Datenstrukturen - Hausübung 03}
\date{ }

\renewcommand*\contentsname{Inhaltsverzeichnis}
\setcounter{secnumdepth}{0}

\begin{document}
	
	\maketitle

	\section*{Gruppenmitglieder}
	
		\begin{itemize}
			\item Emre Berber (2957148)
			\item Christoph Berst (2743394)
			\item Jan Braun (2768531)
		\end{itemize}
	
	\vspace{1.5em}
	\thispagestyle{empty}
	\tableofcontents
	\newpage
	\setcounter{page}{1}
	
	\section{H1}
		
		\subsection{a)}
		
			\includegraphics[width=0.2\linewidth]{picture/H1a/Picture1.png}
			\includegraphics[width=0.2\linewidth]{picture/H1a/Picture2.png}
			\includegraphics[width=0.2\linewidth]{picture/H1a/Picture3.png}
			\includegraphics[width=0.2\linewidth]{picture/H1a/Picture4.png}
			\includegraphics[width=0.2\linewidth]{picture/H1a/Picture5.png}
			\includegraphics[width=0.2\linewidth]{picture/H1a/Picture6.png}
			\includegraphics[width=0.2\linewidth]{picture/H1a/Picture7.png}
			\includegraphics[width=0.2\linewidth]{picture/H1a/Picture8.png}
			\includegraphics[width=0.2\linewidth]{picture/H1a/Picture9.png}
			\includegraphics[width=0.2\linewidth]{picture/H1a/Picture10.png}
			\begin{minipage}{0.2\textwidth}
				\vspace{0pt}
				\includegraphics[width=\linewidth]{picture/H1a/Picture11.png}
			\end{minipage}
			\begin{minipage}{0.8\textwidth}
				\vspace{0pt}
				Wenn dies ein {\color{red} Rot}-{\color{blue} Schwarz}-Baum werden soll, müssten die {\color{red} "1"} und die {\color{red} "6" rot} sein, da die "2" und die "7" über ihnen ansonsten nicht die selbe Anzahl an schwarzen Knoten in ihren Teilbäumen hätten. Somit wären die {\color{blue} "2"} und {\color{blue} "7"} darüber jedoch {\color{blue} schwarz} damit die "Nicht-{\color{red} Rot}-{\color{red} Rot}"-Regel erfüllt bleibt. Nun stellen wir fest, dass egal wie man man "5", "10" und "8" färben möchte einer von ihnen muss noch {\color{blue} schwarz} werden. Das würde jedoch bedeuten, dass der Teilbaum der "5" mindestens {\color{blue} 2 schwarze Knoten}  auf seinen Pfaden hat. Der Teilbaum der "2" jedoch nur einen {\color{blue} schwarzen Knoten} hat. Daraus folgt, dass dies kein {\color{red} Rot}-{\color{blue} Schwarz}-Baum sein kann.
			\end{minipage}
			
		
		\subsection{b)}
			
			qwertz
			
		\subsection{c)}
			
			qwertz
	
	\section{H2}
		
		\subsection{a)}
		
			qwertz
			
		\subsection{b)}
		
			qwertz
			
		\subsection{c)}
		
			qwertz
			
		\subsection{d)}
		
			qwertz
	
	\section{H3}
		
		\subsection{a)}
		
			qwertz
		
		\subsection{b)}
		
			qwertz
		
		\subsection{c)}
		
			qwertz		
			
\end{document}
