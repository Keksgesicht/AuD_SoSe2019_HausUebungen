 \documentclass[ngerman,
 				a4paper,
 				fontsize=12pt]
 				{article}

\usepackage[a4paper, left=22mm, right=22mm, top=22mm, bottom=22mm]{geometry}
\usepackage[hidelinks]{hyperref}
\usepackage[utf8]{inputenc}
\usepackage{pdflscape}
\usepackage{fancyhdr}
\usepackage{graphicx}
\usepackage{amssymb}

\title{Algorithmen und Datenstrukturen - Hausübung 11}
\date{ }

\setcounter{secnumdepth}{0}
\renewcommand*\contentsname{Inhaltsverzeichnis}

% https://tex.stackexchange.com/questions/337/how-to-change-certain-pages-into-landscape-portrait-mode/354#answer-453038
\fancypagestyle{mylandscape}{
	\fancyhf{} %Clears the header/footer
	\fancyfoot{% Footer
		\makebox[\textwidth][r]{% Right
			\rlap{\hspace{.75cm}% Push out of margin by \footskip
				\smash{% Remove vertical height
					\raisebox{4.87in}{% Raise vertically
						\rotatebox{90}{\thepage}}}}}}% Rotate counter-clockwise
	\renewcommand{\headrulewidth}{0pt}% No header rule
	\renewcommand{\footrulewidth}{0pt}% No footer rule
}

\begin{document}
	
	\begin{landscape}
		\maketitle
		
		\section*{Gruppenmitglieder}
		
			\begin{itemize}
				\item Emre Berber (2957148)
				\item Christoph Berst (2743394)
				\item Jan Braun (2768531)
			\end{itemize}
		
		\vspace{1.5em}
		\thispagestyle{empty}
		\tableofcontents
		\newpage
		\setcounter{page}{1}
	\end{landscape}

	\section{H1}
	
		\subsection{a)}
			
			\begin{minipage}[t]{0.45\linewidth}
				\subsubsection{i)}
				
					\begin{tabular}{ | c | c | c | c | }
						\hline
						Iteration & $u$ & $v$ & $Q$
						\\ \hline\hline
						0 & $-$ & $\square$ & \{6\}
						\\ \hline
						1 & 6 & [3,12,14] & \{3,12,14\}
						\\ \hline
						2 & 3 & [2] & \{12,14,2\}
						\\ \hline
						3 & 12 & [19] & \{14,2,19\}
						\\ \hline
						4 & 14 & [11,13,15] & \{2,19,11,13,15\}
						\\ \hline
						5 & 2 & [8] & \{19,11,13,15,8\}
						\\ \hline
						6 & 19 & [9] & \{11,13,15,8,9\}
						\\ \hline
						7 & 11 & [4] & \{13,15,8,9,4\}
						\\ \hline
						8 & 13 & [7,18] & \{15,8,9,4,7,18\}
						\\ \hline
						9 & 15 & $\square$ & \{8,9,4,7,18\}
						\\ \hline
						10 & 8 & $\square$ & \{9,4,7,18\}
						\\ \hline
						11 & 9 & $\square$ & \{4,7,18\}
						\\ \hline
						12 & 4 & $\square$ & \{7,18\}
						\\ \hline
						13 & 7 & $\square$ & \{18\}
						\\ \hline
						14 & 18 & $\square$ & $\emptyset$
						\\ \hline
						15 & $-$ & $\square$ & $\emptyset$
						\\ \hline
						16 & $-$ & $\square$ & $\emptyset$
						\\ \hline
					\end{tabular}
			\end{minipage}
			\begin{minipage}[t]{0.55\linewidth}
				\subsubsection{ii)}
				
					BFS-Baum $G^{6}_{pred}$ \\ \\
					\includegraphics[width=\linewidth]{picture/H1/Picture1.png}
			\end{minipage}
			
		\subsection{b)}
		
			\begin{minipage}[t]{0.62\linewidth}
				\subsubsection{i)}
				
				\begin{tabular}{ | c | c | c | c | }
					\hline
					Knoten & Entdeckungszeit & Abschlusszeit & Vorgängerknoten
					\\ \hline\hline
					1 & 1 & 18 & nil
					\\ \hline
					2 & 19 & 20 & nil
					\\ \hline
					3 & 21 & 22 & nil
					\\ \hline
					4 & 3 & 6 & 5
					\\ \hline
					5 & 2 & 7 & 1
					\\ \hline
					6 & 23 & 24 & nil
					\\ \hline
					7 & 4 & 5 & 4
					\\ \hline
					8 & 25 & 26 & nil
					\\ \hline
					9 & 27 & 28 & nil
					\\ \hline
					10 & 8 & 17 & 1
					\\ \hline
					11 & 11 & 12 & 15
					\\ \hline
					12 & 29 & 30 & nil
					\\ \hline
					13 & 9 & 16 & 10
					\\ \hline
					14 & 31 & 32 & nil
					\\ \hline
					15 & 10 & 13 & 13
					\\ \hline
					16 & 33 & 34 & nil
					\\ \hline
					17 & 35 & 36 & nil
					\\ \hline
					18 & 14 & 15 & 13
					\\ \hline
					19 & 37 & 38 & nil
					\\ \hline
					20 & 39 & 40 & nil
					\\ \hline
				\end{tabular}
			\end{minipage}
			\begin{minipage}[t]{0.38\linewidth}
				\subsubsection{ii)}
				
					topologisch Sortieren: \\
					1,10,13,18,15,15,11,5,4,7 \\ \\
					DFS-Wald $G_{pred}$ \\ \\
					\includegraphics[width=\linewidth]{picture/H1/Picture2.png}
			\end{minipage}	
	
	\newpage
	\thispagestyle{mylandscape}
	\begin{landscape}	
	\section{H2}
	Für die Zeichnungen in dieser Aufgabe gilt, dass gestrichelten Linien sollen nicht Teil des Spannbaums sein.
		\subsection{a)}
			
			\begin{minipage}[l]{0.78\linewidth}
				\begin{center}
					\begin{tabular}{ | c | c | c | c | c | c | c | c | c | c | c | c | c | }
						\hline
						$\{u,v\}$ & $w(\{u,v\})$ & Dazu? & set(a) & set(b) & set(c) & set(d) & set(e) & set(f) & set(g) & set(h) & set(j) & set(k)
						\\ \hline\hline
						$\square$ & $\square$ & $\square$ & \{a\} & \{b\} & \{c\} & \{d\} & \{e\} & \{f\} & \{g\} & \{h\} & \{j\} & \{k\}
						\\ \hline 
						$\{a,c\}$ & 1 & true & \{a,c\} & = & \{a,c\} & = & = & = & = & = & = & =
						\\ \hline
						$\{b,j\}$ & 2 & true & = & \{b,j\} & = & = & = & = & = & = & \{b,j\} & =
						\\ \hline
						$\{c,g\}$ & 2 & true & \{a,c,g\} & = & \{a,c,g\} & = & = & = & \{a,c,g\} & = & = & =
						\\ \hline
						$\{d,h\}$ & 2 & true & = & = & = & \{d,h\} & = & = & = & \{d,h\} & = & =
						\\ \hline
						$\{a,d\}$ & 3 & true & \{a,c,d,g,h\} & = & \multicolumn{2}{|c|}{\{a,c,d,g,h\}} & = & = & \multicolumn{2}{|c|}{\{a,c,d,g,h\}} & = & =
						\\ \hline
						$\{a,g\}$ & 3 & false & = & = & = & = & = & = & = & = & = & =
						\\ \hline
						$\{b,k\}$ & 3 & true & = & \{b,j,k\} & = & = & = & = & = & = & \multicolumn{2}{|c|}{\{b,j,k\}}
						\\ \hline
						$\{a,f\}$ & 4 & true & \{a,c,d,f,g,h\} & = & \multicolumn{2}{|c|}{\{a,c,d,f,g,h\}} & = & \multicolumn{3}{|c|}{\{a,c,d,f,g,h\}} & = & =
						\\ \hline
						$\{f,h\}$ & 4 & false & = & = & = & = & = & = & = & = & = & =
						\\ \hline
						$\{j,k\}$ & 4 & false & = & = & = & = & = & = & = & = & = & =
						\\ \hline
						$\{c,f\}$ & 5 & false & = & = & = & = & = & = & = & = & = & =
						\\ \hline
						$\{e,j\}$ & 5 & true & = & \{b,e,j,k\} & = & = & \{b,e,j,k\} & = & = & = & \multicolumn{2}{|c|}{\{b,e,j,k\}}
						\\ \hline
						$\{c,h\}$ & 7 & false & = & = & = & = & = & = & = & = & = & =
						\\ \hline
						$\{b,e\}$ & 8 & false & = & = & = & = & = & = & = & = & = & =
						\\ \hline
					\end{tabular}
					\\ \vspace{1em}
					\includegraphics[width=0.37\linewidth]{picture/H2/Picture1.png}
					\hfill
					\includegraphics[width=0.40\linewidth]{picture/H2/Picture2.png}
				\end{center}
			\end{minipage}
	
		
		\newpage
		\thispagestyle{mylandscape}
		\subsection{b)}
		
			\begin{tabular}{ | c | c | c | c | c | c | c | c | c | c | c | c | c | c | c | c | c | c | c | c | c | c | }
				\hline
				$a$.k & $b$.k & $c$.k & $d$.k & $e$.k & $f$.k & $g$.k & $h$.k & $j$.k & $k$.k & $l$.k & $m$.k & $n$.k & $p$.k & $r$.k & $u$ & $Q$
				\\ \hline\hline
				$-\infty$ & $\infty$ & $\infty$ & $\infty$ & $\infty$ & $\infty$ & $\infty$ & $\infty$ & $\infty$ & $\infty$ & $\infty$ & $\infty$ & $\infty$ & $\infty$ & $\infty$ & $-$ & $\{a,b,c,d,e,f,g,h,j,k,l,m,n,p,r\}$
				\\ \hline
				$-\infty$ & 3 & 1 & 2 & $\infty$ & 4 & 5 & $\infty$ & $\infty$ & $\infty$ & $\infty$ & $\infty$ & $\infty$ & $\infty$ & $\infty$ & a & $\{c,d,b,f,g,e,h,j,k,l,m,n,p,r\}$
				\\ \hline
				$-\infty$ & 3 & 1 & 2 & $\infty$ & 4 & 2 & 7 & $\infty$ & $\infty$ & $\infty$ & $\infty$ & $\infty$ & $\infty$ & $\infty$ & c & $\{d,g,b,f,h,e,j,k,l,m,n,p,r\}$
				\\ \hline
				$-\infty$ & 1 & 1 & 2 & $\infty$ & 4 & 2 & 3 & $\infty$ & $\infty$ & $\infty$ & $\infty$ & $\infty$ & $\infty$ & $\infty$ & d & $\{b,g,h,f,e,j,k,l,m,n,p,r\}$
				\\ \hline
				$-\infty$ & 1 & 1 & 2 & 8 & 4 & 2 & 3 & $\infty$ & $\infty$ & $\infty$ & $\infty$ & 7 & $\infty$ & $\infty$ & b & $\{g,h,f,n,e,j,k,l,m,p,r\}$
				\\ \hline
				$-\infty$ & 1 & 1 & 2 & 8 & 4 & 2 & 3 & $\infty$ & $\infty$ & $\infty$ & $\infty$ & 7 & $\infty$ & $\infty$ & g & $\{h,f,n,e,j,k,l,m,p,r\}$
				\\ \hline
				$-\infty$ & 1 & 1 & 2 & 8 & 4 & 2 & 3 & $\infty$ & $\infty$ & 7 & $\infty$ & 7 & $\infty$ & $\infty$ & h & $\{f,l,n,e,j,k,m,p,r\}$
				\\ \hline
				$-\infty$ & 1 & 1 & 2 & 8 & 4 & 2 & 3 & $\infty$ & $\infty$ & 7 & 7 & 7 & $\infty$ & $\infty$ & f & $\{l,m,n,e,j,k,p,r\}$
				\\ \hline
				$-\infty$ & 1 & 1 & 2 & 8 & 4 & 2 & 3 & $\infty$ & $\infty$ & 7 & 7 & 7 & $\infty$ & 1 & l & $\{r,m,n,e,j,k,p\}$
				\\ \hline
				$-\infty$ & 1 & 1 & 2 & 8 & 4 & 2 & 3 & $\infty$ & $\infty$ & 7 & 7 & 7 & 3 & 1 & r & $\{p,m,n,e,j,k,\}$
				\\ \hline
				$-\infty$ & 1 & 1 & 2 & 8 & 4 & 2 & 3 & $\infty$ & $\infty$ & 7 & 7 & 7 & 3 & 1 & p & $\{m,n,e,j,k,\}$
				\\ \hline
				$-\infty$ & 1 & 1 & 2 & 8 & 4 & 2 & 3 & $\infty$ & 3 & 7 & 7 & 7 & 3 & 1 & m & $\{k,n,e,j\}$
				\\ \hline
				$-\infty$ & 1 & 1 & 2 & 8 & 4 & 2 & 3 & 2 & 3 & 7 & 7 & 7 & 3 & 1 & k & $\{j,n,e\}$
				\\ \hline
				$-\infty$ & 1 & 1 & 2 & 8 & 4 & 2 & 3 & 2 & 3 & 7 & 7 & 5 & 3 & 1 & j & $\{n,e\}$
				\\ \hline
				$-\infty$ & 1 & 1 & 2 & 8 & 4 & 2 & 3 & 2 & 3 & 7 & 7 & 5 & 3 & 1 & n & $\{e\}$
				\\ \hline
				$-\infty$ & 1 & 1 & 2 & 8 & 4 & 2 & 3 & 2 & 3 & 7 & 7 & 5 & 3 & 1 & e & $\emptyset$
				\\ \hline
			\end{tabular}
			\\ \\ \\
			\begin{tabular}{ | c | c | c | c | c | c | c | c | c | c | c | c | c | c | c | c | c | c | c | c | }
				\hline
				$a$.p & $b$.p & $c$.p & $d$.p & $e$.p & $f$.p & $g$.p & $h$.p & $j$.p & $k$.p & $l$.p & $m$.p & $n$.p & $p$.p & $r$.p & $u$ & $Q$
				\\ \hline\hline
				nil & nil & nil & nil & nil & nil & nil & nil & nil & nil & nil & nil & nil & nil & nil & $-$ & $\{a,b,c,d,e,f,g,h,j,k,l,m,n,p,r\}$
				\\ \hline
				nil & a & a & a & nil & a & a & nil & nil & nil & nil & nil & nil & nil & nil & a & $\{c,d,b,f,g,e,h,j,k,l,m,n,p,r\}$
				\\ \hline
				nil & a & a & a & nil & a & c & c & nil & nil & nil & nil & nil & nil & nil & c & $\{d,g,b,f,h,e,j,k,l,m,n,p,r\}$
				\\ \hline
				nil & d & a & a & nil & a & c & d & nil & nil & nil & nil & nil & nil & nil & d & $\{b,g,h,f,e,j,k,l,m,n,p,r\}$
				\\ \hline
				nil & d & a & a & b & a & c & d & nil & nil & nil & nil & b & nil & nil & b & $\{g,h,f,n,e,j,k,l,m,p,r\}$
				\\ \hline
				nil & d & a & a & b & a & c & d & nil & nil & nil & nil & b & nil & nil & g & $\{h,f,n,e,j,k,l,m,p,r\}$
				\\ \hline
				nil & d & a & a & b & a & c & d & nil & nil & h & nil & b & nil & nil & h & $\{f,l,n,e,j,k,m,p,r\}$
				\\ \hline
				nil & d & a & a & b & a & c & d & nil & nil & h & f & b & nil & nil & f & $\{n,l,m,e,j,k,p,r\}$
				\\ \hline
				nil & d & a & a & b & a & c & d & nil & nil & h & f & b & nil & l & l & $\{r,m,n,e,j,k,p\}$
				\\ \hline
				nil & d & a & a & b & a & c & d & nil & nil & h & f & b & r & l & r & $\{p,m,n,e,j,k\}$
				\\ \hline
				nil & d & a & a & b & a & c & d & nil & nil & h & f & b & r & l & p & $\{m,n,e,j,k\}$
				\\ \hline
				nil & d & a & a & b & a & c & d & nil & m & h & f & b & r & l & m & $\{k,n,e,j\}$
				\\ \hline
				nil & d & a & a & b & a & c & d & k & m & h & f & b & r & l & k & $\{j,n,e\}$
				\\ \hline
				nil & d & a & a & b & a & c & d & k & m & h & f & j & r & l & j & $\{n,e\}$
				\\ \hline
				nil & d & a & a & b & a & c & d & k & m & h & f & j & r & l & n & $\{e\}$
				\\ \hline
				nil & d & a & a & b & a & c & d & k & m & h & f & j & r & l & e & $\emptyset$
				\\ \hline
			\end{tabular}
	\end{landscape}
			
			\newpage
			\includegraphics[width=0.76\linewidth]{picture/H2/Picture3.png}
			
	\section{H3}
	
		\subsection{a)}
		
			\textit{"to be filled in!"}
	
		\vfill	
		\subsection{b)}
	
			\begin{minipage}[t]{0.46\textwidth}
				\vspace{0pt}
				\fbox{\begin{minipage}{\textwidth} % box
						\ttfamily
						\begin{tabbing}
							\hspace*{2em}\= \hspace*{2em}\= \hspace*{2em}\= \hspace*{2em}\= \hspace*{2em}\= \kill % set the tabbings
							GraphTriangles(G,V,E) \\
							1	\>	c=0 \\
							2	\>	FOR u=0 TO V.length$-$1 \\
							3	\>	\>	FOR v=u+1 TO adj(G,u).length$-$1 \\
							5	\>	\>	\>	IF adj(G,u,V[v]) == 0 \\
							6	\>	\>	\>	\>	CONTINUE \\
							7	\>	\>	\>	FOR w=v+1 TO adj(G,u).length$-$1 \\
							8	\>	\>	\>	\>	IF adj(G,u,V[w]) == 0 \\
							9	\>	\>	\>	\>	\>	CONTINUE \\
							10	\>	\>	\>	\>	IF $\{V[v],V[w]\} \in$ E \\
							11	\>	\>	\>	\>	\>	c++ \\
							12	\>	return c
						\end{tabbing}
				\end{minipage}}
			\end{minipage}
			\begin{minipage}[t]{0.54\textwidth}
				\vspace{0pt}
				\begin{itemize}
					\item V ist die Liste der Knoten in G und E ist die Liste der Kanten in G. \{v,w\} $\in$ E ist eine ungerichte Kante zwischen v und w.
					\item adj(G,v,w) gibt das Gewicht der Kante zwischen \mbox{v und w} zurück. Mit v,w $\in$ V
					\item adj(G,u).length$-$1 sollte identisch zu V.length$-$1 sein, da es sich bei adj(G,u) im die Zeile des Knotens u in der Adjazenzmatrize von G handelt, welche mit dem Wert 0 angibt das keine Verbindung \mbox{von u zu V[i]} besteht und ansonsten das Gewicht der Kante zwischen den Knoten u und V[i] enthält.
				\end{itemize}
				
			\end{minipage}
			
		\subsection{c)}
			
			\underline{Komplexität:} \\
			Wir druchlaufen n Knoten mit u, maximal n Kanten von u mit v und maximal weitere n Kanten \mbox{von u mit w.}
			Dies macht dann n $\cdot$ n $\cdot$ n = $n^{3}$
			\\ \\
			\underline{Korrektheit:} \\
			\textit{"to be filled in!"}			
	
\end{document}
