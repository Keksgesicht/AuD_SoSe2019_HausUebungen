 \documentclass[ngerman,
 				a4paper,
 				fontsize=12pt]
 				{article}

\usepackage[a4paper, left=22mm, right=22mm, top=22mm, bottom=22mm]{geometry}
\usepackage[hidelinks]{hyperref}
\usepackage[utf8]{inputenc}

\title{Algorithmen und Datenstrukturen - Hausübung 11}
\date{ }

\renewcommand*\contentsname{Inhaltsverzeichnis}
\setcounter{secnumdepth}{0}

\hyphenation{Wer-ten}

\begin{document}
	
	\maketitle

	\section*{Gruppenmitglieder}
	
		\begin{itemize}
			\item Emre Berber (2957148)
			\item Christoph Berst (2743394)
			\item Jan Braun (2768531)
		\end{itemize}
	
	\vspace{1.5em}
	\thispagestyle{empty}
	\tableofcontents
	\newpage
	\setcounter{page}{1}
	
	\section{H1}
	
		\subsection{a)}
			
			\subsubsection{i)}
			
				\textit{"to be filled in!"}
			
			\vfill
			\subsubsection{ii)}
			
				\textit{"to be filled in!"}
		
		\vfill	
		\subsection{b)}
		
			\subsubsection{i)}
		
				\textit{"to be filled in!"}
		
			\vfill
			\subsubsection{ii)}
			
				\textit{"to be filled in!"}
	
	\vfill			
	\section{H2}
	
		\subsection{a)}
		
			\textit{"to be filled in!"}
	
		\vfill	
		\subsection{b)}
		
			\textit{"to be filled in!"}
			
	\vfill			
	\section{H3}
	
		\subsection{a)}
		
			\textit{"to be filled in!"}
	
		\vfill	
		\subsection{b)}
	
			\textit{"to be filled in!"}
		
		\vfill	
		\subsection{c)}
		
			\textit{"to be filled in!"}
	
\end{document}
