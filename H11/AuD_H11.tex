 \documentclass[ngerman,
 				a4paper,
 				fontsize=12pt]
 				{article}

\usepackage[a4paper, landscape=true, left=22mm, right=22mm, top=22mm, bottom=22mm]{geometry}
\usepackage[hidelinks]{hyperref}
\usepackage[utf8]{inputenc}
\usepackage{amssymb}

\title{Algorithmen und Datenstrukturen - Hausübung 11}
\date{ }

\renewcommand*\contentsname{Inhaltsverzeichnis}
\setcounter{secnumdepth}{0}

\begin{document}
	
	\maketitle

	\section*{Gruppenmitglieder}
	
		\begin{itemize}
			\item Emre Berber (2957148)
			\item Christoph Berst (2743394)
			\item Jan Braun (2768531)
		\end{itemize}
	
	\vspace{1.5em}
	\thispagestyle{empty}
	\tableofcontents
	\newpage
	\setcounter{page}{1}
	
	\section{H1}
	
		\subsection{a)}
			
			\subsubsection{i)}
			
				\textit{"to be filled in!"}
			
			\vfill
			\subsubsection{ii)}
			
				\textit{"to be filled in!"}
		
		\vfill	
		\subsection{b)}
		
			\subsubsection{i)}
		
				\textit{"to be filled in!"}
		
			\vfill
			\subsubsection{ii)}
			
				\textit{"to be filled in!"}
	
	\vfill			
	\section{H2}
	
		\subsection{a)}
			
			\begin{tabular}{ | c | c | c | c | c | c | c | c | c | c | c | c | c | }
				\hline
				$\{u,v\}$ & $w(\{u,v\})$ & Dazu? & set(a) & set(b) & set(c) & set(d) & set(e) & set(f) & set(g) & set(h) & set(j) & set(k)
				\\ \hline
				$\square$ & $\square$ & $\square$ & \{a\} & \{b\} & \{c\} & \{d\} & \{e\} & \{f\} & \{g\} & \{h\} & \{j\} & \{k\}
				\\ \hline 
				$\{a,c\}$ & 1 & true & \{a,c\} & = & \{a,c\} & = & = & = & = & = & = & =
				\\ \hline
				$\{b,j\}$ & 2 & true & = & \{b,j\} & = & = & = & = & = & = & \{b,j\} & =
				\\ \hline
				$\{c,g\}$ & 2 & true & \{a,c,g\} & = & \{a,c,g\} & = & = & = & \{a,c,g\} & = & = & =
				\\ \hline
				$\{d,h\}$ & 2 & true & = & = & = & \{d,h\} & = & = & = & \{d,h\} & = & =
				\\ \hline
				$\{a,d\}$ & 3 & true & \{a,c,d,g,h\} & = & \multicolumn{2}{|c|}{\{a,c,d,g,h\}} & = & = & \multicolumn{2}{|c|}{\{a,c,d,g,h\}} & = & =
				\\ \hline
				$\{a,g\}$ & 3 & false & = & = & = & = & = & = & = & = & = & =
				\\ \hline
				$\{b,k\}$ & 3 & true & = & \{b,j,k\} & = & = & = & = & = & = & \multicolumn{2}{|c|}{\{b,j,k\}}
				\\ \hline
				$\{a,f\}$ & 4 & true & \{a,c,d,f,g,h\} & = & \multicolumn{2}{|c|}{\{a,c,d,f,g,h\}} & = & \multicolumn{3}{|c|}{\{a,c,d,f,g,h\}} & = & =
				\\ \hline
				$\{f,h\}$ & 4 & false & = & = & = & = & = & = & = & = & = & =
				\\ \hline
				$\{j,k\}$ & 4 & false & = & = & = & = & = & = & = & = & = & =
				\\ \hline
				$\{c,f\}$ & 5 & false & = & = & = & = & = & = & = & = & = & =
				\\ \hline
				$\{e,j\}$ & 5 & true & = & \{b,e,j,k\} & = & = & \{b,e,j,k\} & = & = & = & \multicolumn{2}{|c|}{\{b,e,j,k\}}
				\\ \hline
				$\{c,h\}$ & 7 & false & = & = & = & = & = & = & = & = & = & =
				\\ \hline
				$\{b,e\}$ & 8 & false & = & = & = & = & = & = & = & = & = & =
				\\ \hline
			\end{tabular}
	
		\vfill	
		\subsection{b)}
		
			\begin{tabular}{ | c | c | c | c | c | c | c | c | c | c | c | c | c | c | c | c | c | c | c | }
				\hline
				$a$.k & $b$.k & $c$.k & $d$.k & $e$.k & $f$.k & $g$.k & $h$.k & $j$.k & $k$.k & $l$.k & $m$.k & $n$.k & $p$.k & $r$.k & $u$
				\\ \hline
				$-\infty$ & $\infty$ & $\infty$ & $\infty$ & $\infty$ & $\infty$ & $\infty$ & $\infty$ & $\infty$ & $\infty$ & $\infty$ & $\infty$ & $\infty$ & $\infty$ & $\infty$ & $-$
				\\ \hline
			\end{tabular}
			\\ \\ \\
			\begin{tabular}{ | c | c | c | c | c | c | c | c | c | c | c | c | c | c | c | c | c | c | c | }
				\hline
				$a$.p & $b$.p & $c$.p & $d$.p & $e$.p & $f$.p & $g$.p & $h$.p & $j$.p & $k$.p & $l$.p & $m$.p & $n$.p & $p$.p & $r$.p & $Q$
				\\ \hline
				nil & nil & nil & nil & nil & nil & nil & nil & nil & nil & nil & nil & nil & nil & nil & $\{a,b,c,d,e,f,g,h,j,k,l,m,n,p,r\}$
				\\ \hline
			\end{tabular}
			
	\vfill			
	\section{H3}
	
		\subsection{a)}
		
			\textit{"to be filled in!"}
	
		\vfill	
		\subsection{b)}
	
			\textit{"to be filled in!"}
		
		\vfill	
		\subsection{c)}
		
			\textit{"to be filled in!"}
	
\end{document}
