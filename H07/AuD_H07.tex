 \documentclass[ngerman,
 				a4paper,
 				fontsize=12pt]
 				{article}

\usepackage[a4paper, left=22mm, right=22mm, top=22mm, bottom=22mm]{geometry}
\usepackage[hidelinks]{hyperref}
\usepackage[utf8]{inputenc}
\usepackage{graphicx}
\usepackage{xcolor}

\title{Algorithmen und Datenstrukturen - Hausübung 07}
\date{ }

\renewcommand*\contentsname{Inhaltsverzeichnis}
\setcounter{secnumdepth}{0}

\begin{document}
	
	\maketitle

	\section*{Gruppenmitglieder}
	
		\begin{itemize}
			\item Emre Berber (2957148)
			\item Christoph Berst (2743394)
			\item Jan Braun (2768531)
		\end{itemize}
	
	\vspace{1.5em}
	\thispagestyle{empty}
	\tableofcontents
	\newpage
	\setcounter{page}{1}
	
	\section{H1}
		
		\subsection{a)}
			
			\includegraphics[width=0.12\linewidth]{picture/H1a/Picture1.png}
			\hspace{0.02\linewidth}
			\includegraphics[width=0.12\linewidth]{picture/H1a/Picture2.png}
			\hspace{0.05\linewidth}
			\includegraphics[width=0.18\linewidth]{picture/H1a/Picture3.png}
			\hspace{0.02\linewidth}
			\includegraphics[width=0.20\linewidth]{picture/H1a/Picture4.png}
			\hspace{0.05\linewidth}
			\includegraphics[width=0.15\linewidth]{picture/H1a/Picture5.png}
			\\ \\
			\includegraphics[width=0.25\linewidth]{picture/H1a/Picture6.png}
			\hspace{0.05\linewidth}
			\includegraphics[width=0.25\linewidth]{picture/H1a/Picture7.png}
			\hspace{0.05\linewidth}
			\includegraphics[width=0.33\linewidth]{picture/H1a/Picture8.png}
			\\ \\
			\includegraphics[width=0.25\linewidth]{picture/H1a/Picture9.png}
			\hspace{0.05\linewidth}
			\includegraphics[width=0.33\linewidth]{picture/H1a/Picture10.png}
			\hspace{0.05\linewidth}
			\includegraphics[width=0.27\linewidth]{picture/H1a/Picture11.png}
			\\ \\
			{\color{gray} grau:} Elterknoten y
		
		\vfill	
		\subsection{b)}
			\begin{minipage}[t]{0.42\linewidth}
				\includegraphics[width=\linewidth]{picture/H1b/Picture1.png}
				{\color{blue} blau:} gesuchter Knoten
			\end{minipage}
			\hfill
			\begin{minipage}[t]{0.42\linewidth}
				\includegraphics[width=\linewidth]{picture/H1b/Picture2.png}
				{\color{orange} orange:} "größter" Knoten y aus L
			\end{minipage}
	
	\newpage		
	\section{H2}
		
		\subsection{a)}
			
			\includegraphics[width=0.32\linewidth]{picture/H2a/Picture1.png}
			\hspace{0.03\linewidth}
			\includegraphics[width=0.32\linewidth]{picture/H2a/Picture2.png}
			\hspace{0.03\linewidth}
			\includegraphics[width=0.32\linewidth]{picture/H2a/Picture3.png}
			\\ \\
			\includegraphics[width=0.47\linewidth]{picture/H2a/Picture4.png}
			\hspace{0.06\linewidth}
			\includegraphics[width=0.47\linewidth]{picture/H2a/Picture5.png}
			\\ \\
			Nach HeapSort(H.A) ist der Baum leer und ein absteigend sortiertes Array ist entstanden. 			
		
		\vfill		
		\subsection{b)}
			
			\begin{itemize}
				\item BuildHeap(H.A) mit A = \{64, 16, 1, 4, 2\}
				\item ExtractMax(H)
				\item Insert(H,32)
				\item ExtractMax(H)
				\item Insert(H,8) 
			\end{itemize}
		
		\subsection{c)}
				
			\begin{minipage}[t]{0.71\textwidth}
				\vspace{0pt}
				\fbox{\begin{minipage}{\textwidth} % box
						\ttfamily
						\begin{tabbing}
							\hspace*{2em}\= \hspace*{2em}\= \hspace*{2em}\= \hspace*{2em}\= \hspace*{2em}\= \kill % set the tabbings
							delete(H,i) \\
							1	\>	IF isEmpty(H) THEN \\
							2	\>	\>	ERROR 'underflow' \\
							3	\>	IF i == 0 THEN \\
							4	\>	\>	RETURN extractMax(H) \\
							5	\>	IF i == H.size-1 THEN \\
							6	\>	\>	result = H.A \\
							7	\>	\>	H.size = H.size-1 \\
							8	\>	ELSE \\
							9	\>	\>	result = H.A \\
							10	\>	\>	H.A = H.A[H.size-1] \\
							11	\>	\>	H.size = H.size-1 \\
							12	\>	\>	heapify (H,i) \\
							13	\>	RETURN result
						\end{tabbing}
				\end{minipage}}
			\end{minipage}			
	
	\vfill
	\section{H3}
		
		\subsection{a)}
			
			\begin{minipage}[t]{0.71\textwidth}
				\vspace{0pt}
				\fbox{\begin{minipage}{\textwidth} % box
						\ttfamily
						\begin{tabbing}
							\hspace*{2em}\= \hspace*{2em}\= \hspace*{2em}\= \hspace*{2em}\= \hspace*{2em}\= \kill % set the tabbings
							minPalindrom(w) \\
							1	\>	count = 0 \\
							2	\>	FOR i=0 TO w.length $-$ 1 \\
							3	\>	\>	j = w.length $-$ 1 \\
							4 	\>	\>	WHILE true DO \\
							5	\>	\>	\>	WHILE(w[i] $\neq$ w[j]) DO \\
							6	\>	\>	\>	\>	j-- \\
							7	\>	\>	\>	IF i == j THEN \\
							8	\>	\>	\>	\>	BREAK \\
							9	\>	\>	\>	IF checkPalindrom(w, i, j) THEN \\
							10	\>	\>	\>	\>	i = j // next iteration of FOR should be i=j+1 \\
							11	\>	\>	\>	\>	BREAK \\
							13	\>	\>	count++ \\
							14	\>	RETURN count
						\end{tabbing}
				\end{minipage}}
			\end{minipage}
			\begin{minipage}[t]{0.29\textwidth}
				\vspace{0pt}
				\fbox{\begin{minipage}{\textwidth} % box
						\ttfamily
						\begin{tabbing}
							\hspace*{2em}\= \hspace*{2em}\= \kill % set the tabbings
							checkPalindrom(w, i, j) \\
							1	\>	WHILE w[i] == w[j] DO \\
							2	\>	\>	i++ \\
							3	\>	\>	j-- \\
							4	\>	IF i < j THEN \\
							5	\>	\> RETURN false \\
							6	\>	ELSE \\
							7	\>	\> RETURN true
						\end{tabbing}
				\end{minipage}}
			\end{minipage}
			\vspace{1ex} \\
			Mit i iterieren wir von links nach rechts über das Wort w und versuchen dabei jeweils mit Hilfe von j das größtmögliche Palindrom, das mit an i beginnt, zu finden. Wenn wir immer die größtmöglichen Palindrome finden, finden wir gleichzeitig auch die geringste Anzahl an Palindromen, da die Vorherigen den meisten Platz für die Folgenden wegnehmen.
		
		\vfill	
		\subsection{b)}
		
			\textit{"to be filled in"}
			
\end{document}
