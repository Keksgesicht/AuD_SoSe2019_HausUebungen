 \documentclass[ngerman,
 				a4paper,
 				fontsize=12pt]
 				{article}

\usepackage[a4paper, left=22mm, right=22mm, top=22mm, bottom=22mm]{geometry}
\usepackage[hidelinks]{hyperref}
\usepackage[utf8]{inputenc}
\usepackage{graphicx}
\usepackage{xcolor}

\title{Algorithmen und Datenstrukturen - Hausübung 09}
\date{ }

\setlength{\parindent}{0pt}
\renewcommand*\contentsname{Inhaltsverzeichnis}
\setcounter{secnumdepth}{0}

\begin{document}
	
	\maketitle

	\section*{Gruppenmitglieder}
	
		\begin{itemize}
			\item Emre Berber (2957148)
			\item Christoph Berst (2743394)
			\item Jan Braun (2768531)
		\end{itemize}
	
	\vspace{1.5em}
	\thispagestyle{empty}
	\tableofcontents
	\newpage
	\setcounter{page}{1}
	
	\section{H1}
		
		\subsection{a)}
			
			\vspace{-13mm}
			\includegraphics[width=0.12\linewidth]{picture/H1a/Picture1.png}
			\hspace{0.02\linewidth}
			\includegraphics[width=0.15\linewidth]{picture/H1a/Picture2.png}
			\hspace{0.05\linewidth}
			\includegraphics[width=0.23\linewidth]{picture/H1a/Picture3.png}
			\hspace{0.05\linewidth}
			\includegraphics[width=0.32\linewidth]{picture/H1a/Picture4.png}
			\\ \\
			\includegraphics[width=0.42\linewidth]{picture/H1a/Picture5.png}
			\hfill
			\includegraphics[width=0.50\linewidth]{picture/H1a/Picture6.png}
			\\ \\
			\includegraphics[width=0.45\linewidth]{picture/H1a/Picture7.png}
			\hfill
			\includegraphics[width=0.52\linewidth]{picture/H1a/Picture8.png}
			\\ \\
			\includegraphics[width=0.48\linewidth]{picture/H1a/Picture9.png}
			\hfill
			\includegraphics[width=0.48\linewidth]{picture/H1a/Picture10.png}
			\\ \\
			\includegraphics[width=0.54\linewidth]{picture/H1a/Picture11.png}
			
		\vfill
		\subsection{b)}
		
			\includegraphics[width=0.5125\linewidth]{picture/H1b/Picture1.png}
			\hfill
			\includegraphics[width=0.46\linewidth]{picture/H1b/Picture2.png}
		
		\vfill	
		\subsection{c)}
		
			\textit{"to be filled in"}
			
		\vfill
		\subsection{d)}
		
			\textit{"to be filled in"}
		
	\vfill
	\section{H2}
		
		\textit{"to be filled in"}
	
	\vfill
	\section{H3}
	
		\subsection{a)}
			
			\subsubsection{i)}
			
				\textit{"to be filled in"}
			
			\vfill	
			\subsubsection{ii)}
			
				\textit{"to be filled in"}
				
			\vfill
			\subsubsection{iii)}
			
				\textit{"to be filled in"}
		
		\vfill
		\subsection{b)}
	
			\subsubsection{i)}
			
				\textit{"to be filled in"}
			
			\vfill	
			\subsubsection{ii)}
			
				\textit{"to be filled in"}
			
			
\end{document}
